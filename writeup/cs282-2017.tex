%%
%% Copyright 2007, 2008, 2009 Elsevier Ltd
%%
%% This file is part of the 'Elsarticle Bundle'.
%% ---------------------------------------------
%%
%% It may be distributed under the conditions of the LaTeX Project Public
%% License, either version 1.2 of this license or (at your option) any
%% later version.  The latest version of this license is in
%%    http://www.latex-project.org/lppl.txt
%% and version 1.2 or later is part of all distributions of LaTeX
%% version 1999/12/01 or later.
%%
%% The list of all files belonging to the 'Elsarticle Bundle' is
%% given in the file `manifest.txt'.
%%

%% Template article for Elsevier's document class `elsarticle'
%% with harvard style bibliographic references
%% SP 2008/03/01
%%
%%
%%
%% $Id: elsarticle-template-harv.tex 4 2009-10-24 08:22:58Z rishi $
%%
%%


%% WHAT WE WERE USING \documentclass[final,authoryear,11pt,times]{elsarticle}

%% Use the option review to obtain double line spacing
%% \documentclass[authoryear,preprint,review,12pt]{elsarticle}

%% Use the options 1p,twocolumn; 3p; 3p,twocolumn; 5p; or 5p,twocolumn
%% for a journal layout:
%% \documentclass[final,authoryear,1p,times]{elsarticle}
%%\documentclass[final,authoryear,1p,times,twocolumn]{elsarticle}
%%\documentclass[final,authoryear,3p,times]{elsarticle}
%%\documentclass[final,authoryear,3p,times,twocolumn]{elsarticle}
\documentclass[final,authoryear,5p,times,twocolumn]{elsarticle}

%% if you use PostScript figures in your article
%% use the graphics package for simple commands
%% \usepackage{graphics}
%% or use the graphicx package for more complicated commands
%% \usepackage{graphicx}
%% or use the epsfig package if you prefer to use the old commands
%% \usepackage{epsfig}

%% The amssymb package provides various useful mathematical symbols
\usepackage{amssymb}
\usepackage{amsmath}

%%\usepackage[margin=1.25in]{geometry}


\usepackage{todonotes}
\usepackage{epigraph}
\usepackage{graphicx}
\usepackage{setspace}
\usepackage{hyperref}
\usepackage{soul}
\usepackage{color}
\usepackage{cleveref}
\crefname{section}{�}{��}
\Crefname{section}{�}{��}
\renewcommand{\sectionautorefname}{\S}
\renewcommand{\subsectionautorefname}{\S}
\onehalfspacing

%% The amsthm package provides extended theorem environments
%% \usepackage{amsthm}

%% The lineno packages adds line numbers. Start line numbering with
%% \begin{linenumbers}, end it with \end{linenumbers}. Or switch it on
%% for the whole article with \linenumbers after \end{frontmatter}.
%% \usepackage{lineno}

%% natbib.sty is loaded by default. However, natbib options can be
%% provided with \biboptions{...} command. Following options are
%% valid:

%%   round  -  round parentheses are used (default)
%%   square -  square brackets are used   [option]
%%   curly  -  curly braces are used      {option}
%%   angle  -  angle brackets are used    <option>
%%   semicolon  -  multiple citations separated by semi-colon (default)
%%   colon  - same as semicolon, an earlier confusion
%%   comma  -  separated by comma
%%   authoryear - selects author-year citations (default)
%%   numbers-  selects numerical citations
%%   super  -  numerical citations as superscripts
%%   sort   -  sorts multiple citations according to order in ref. list
%%   sort&compress   -  like sort, but also compresses numerical citations
%%   compress - compresses without sorting
%%   longnamesfirst  -  makes first citation full author list
%%
%% \biboptions{longnamesfirst,comma}

% \biboptions{}

\journal{CS282, Fall 2017 - Final Project Report, Wihl, Ding, Peng}

\begin{document}

\begin{frontmatter}

%% Title, authors and addresses

%% use the tnoteref command within \title for footnotes;
%% use the tnotetext command for the associated footnote;
%% use the fnref command within \author or \address for footnotes;
%% use the fntext command for the associated footnote;
%% use the corref command within \author for corresponding author footnotes;
%% use the cortext command for the associated footnote;
%% use the ead command for the email address,
%% and the form \ead[url] for the home page:
%%
%% \title{Title\tnoteref{label1}}
%% \tnotetext[label1]{}
%% \author{Name\corref{cor1}\fnref{label2}}
%% \ead{email address}
%% \ead[url]{home page}
%% \fntext[label2]{}
%% \cortext[cor1]{}
%% \address{Address\fnref{label3}}
%% \fntext[label3]{}

\title{Generative Models for Sepsis}

%% use optional labels to link authors explicitly to addresses:
%% \author[label1,label2]{<author name>}
%% \address[label1]{<address>}
%% \address[label2]{<address>}

\author{{\rm David Wihl, Xuefeng Peng, Yi Ding}\\ Harvard University}
\address{\normalsize\{davidwihl,xpeng\}@g.harvard.edu, \normalsize{yid095}@mail.harvard.edu}

\begin{abstract}

Abstract goes here

\end{abstract}

\end{frontmatter}

\section*{Introduction}
\label{sec:introduction}

Intro goes here

\section{Related Work}
\label{sec:related}

Related Work goes here


\subsection{Classic Physiological Simulators}

stuff 

\subsection{Generative Models in RL}

stuff

\subsection{Sepsis RL}

more stuff

\section{System Design}
\label{sec:design}

Design


\section{Validation}
\label{sec:validation}

How did we validate?


\section{Advanced RL Implementation}
\label{sec:advanced}

\section{Discussion}
\label{sec:discuss}

discuss stuff

\subsection{Validation Methodology}
\label{subsec:eval}
 

\subsection{Limitations}
\label{subsec:limitations}

limitations


\section {Future Work}
\label{sec:future}

Crystal ball goes here...


\section*{Conclusion}
  
Conclusion
 
\section*{Acknowledgements}

Special thanks to everyone

\appendix

\nocite{*}


% NOTE: Replaced the class-provided elsarticle-num-names with elsarticle-num-names-alpha
%       to enable sorting of references.
% See https://tex.stackexchange.com/questions/297283/order-references-alphabetically-with-elsarticle-num-names-bst
% and https://github.com/erelsgl/erelsgl.github.io/blob/master/papers/elsarticle-num-names-alpha.bst
%
% To revert to the prior bibliographystyle, change the following line back to:
%             \bibliographystyle{elsarticle-num-names}

\bibliographystyle{elsarticle-num-names-alpha}
\bibliography{cs282-bib}


\end{document}


%% The Appendices part is started with the command \appendix;
%% appendix sections are then done as normal sections
%% \appendix

%% \section{}
%% \label{}
